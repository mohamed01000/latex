\documentclass{article}




\usepackage[utf8]{inputenc}
\usepackage[T1]{fontenc}

\usepackage{hyperref}
\usepackage{xcolor}
\usepackage{listings}


\title{TP GIT}
\author{M.S}
\date {23/03/2020}
\begin {document}
\maketitle{}
\section {Git les bases}
\definecolor{mGreen}{rgb}{0,0.6,0}
\definecolor{mGray}{rgb}{0.5,0.5,0.5}
\definecolor{mPurple}{rgb}{0.58,0,0.82}
\definecolor{backgroundColour}{rgb}{0.95,0.95,0.92}

\lstdefinestyle{CStyle}{
    backgroundcolor=\color{backgroundColour},   
    commentstyle=\color{mGreen},
    keywordstyle=\color{magenta},
    numberstyle=\tiny\color{mGray},
    stringstyle=\color{mPurple},
    basicstyle=\footnotesize,
    breakatwhitespace=false,         
    breaklines=true,                 
    captionpos=b,                    
    keepspaces=true,                 
    numbers=left,                    
    numbersep=5pt,                  
    showspaces=false,                
    showstringspaces=false,
    showtabs=false,                  
    tabsize=2,
    language=C
}

% faire le plan 
Git est le sytème de gestion de version décentralisé open source qui facilite les activités GitHub sur votre ordinateur.
Cet aide-mémoire permet un accès rapide aux instructions des commandes Git les plus utilisées.	

\subsection{Créer son premier dépôt}
La première partie
\\ 
Pré-requis il faut un logiciel Git Bash  
il faut se placer dans le répertoire que l'on souhaite utiliser. 
\\ 
On saisie dans l'invite de commande 
\textit{git init} 


\begin{lstlisting}[style=CStyle]

Put the code here. 
// Mon premier fichier en C 
// 
// 

#include <stdio.h>

int main(void)
{

	printf("Hello world");
}
\end{lstlisting}

\begin{lstlisting}[style=CStyle]
git init 
\end{lstlisting}


\subsection{Partie 2}

Quelques commandes de bases pour commencer: 
\\
\\
\textit{git init} crée un nouveau dépôt
\\ 
\textit{git add} ajout un fichier au dépôt

\section {Commande de bases}
\subsection{Partie 1}

\end{document} 


